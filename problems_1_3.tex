\documentclass[11pt]{article}
\usepackage{graphicx} % Required for inserting images
\usepackage[english]{babel}
\usepackage{amsmath}
\usepackage{graphicx}
\usepackage[colorlinks=true, allcolors=blue]{hyperref}
\usepackage{epstopdf}
\usepackage{cleveref}
\usepackage{float}
\usepackage[a4paper,top=2cm,bottom=2cm,left=3cm,right=3cm,marginparwidth=1.75cm]{geometry}
% -------------------------------------------------------------
\newcommand{\dd}{\,\mathrm{d}}
\newcommand{\R}{\mathbb{R}}
\newcommand{\RR}
{\mathcal{R}_0}

\title{Mathematical Model of a Dynamical System: Spread of Disease}
\date{\today}

\begin{document}
\maketitle

\section{Problem-1: The SI Model without Recovery}

Consider a closed, well--mixed population consisting of susceptibles-\(S(t)\) and infecteds-\(I(t)\) at time \(t\in\R_{\ge 0}\). The total population

\[
N\;=\;S(t)+I(t)\quad(\forall t)
\]
is conserved.

\subsection*{(a) Model formulation}
Let 
\(\beta>0\) denote the \emph{effective contact rate}: the average number  of potentially infectious contacts an individual makes per unit timie, multiplied by the probability that such a contact transmits infection. Under homogeneous mixing the incidence of new infections is proportional to the number of susceptible--infected pairs \emph{per capita}, leading to the \emph{SI}\ system
\begin{align}
    \frac{\dd S}{\dd t} &= -\frac{\beta}{N}\, S I, \\[4pt]
    \frac{\dd I}{\dd t} &= \frac{\beta}{N}\, S I.
\label{eq:SI}
\end{align}

\noindent
The factor \(S/N\) represents the probability that a contact made by an infectious individual is with a susceptible person, while \(\beta I\) is the expected number of effective (potentially infectious) contacts generated per unit time by the entire infected class. Their product,
\(\beta \,(S/N)\, I\), therefore equals the \emph{incidence}: the instantaneous rate at which new infections arise.

\subsection*{(b) Biological meaning of \(\beta\)}
We can write
\[
\beta = c\,p,
\]
where \(c\) is the mean contact rate (contacts per individual per unit time) and \(p\) is the probability a contact transmits infection. % Add typical values for betta with ref.s 
Biologically, \(\beta\) represents the \emph{effective contact rate}: the average number of successful transmissions generated by one infectious individual per unit time. 

\subsection*{(c) Reduction and long--term behaviour}
Because \(N\) is constant, \(S = N-I\). Substituting into~\eqref{eq:SI} give the logistic equation
\begin{equation}
\frac{\dd I}{\dd t}= \beta I\left(1-\frac{I}{N}\right).
\label{eq:logistic}
\end{equation}
With initial condition $I(0)=I_0>0$ the solution is
\[
I(t)=\frac{N}{1+\bigl(\tfrac{N}{I_0}-1\bigr)e^{-\beta t}}.
\]
Hence $\displaystyle\lim_{t\to\infty}I(t)=N$ and $S(t)=N-I(t)\to0$.  \\Biologically, without recovery every individual is eventually infected.

% -------------------------------------------------------------
\section{Problem~2: Pure Recovery (No New Infections)}
Assume the entire population is initially infected ($I(0)=N$) and that each infected individual recovers independently at rate $\gamma>0$.

\subsection*{(a) Model}
The expected number of infecteds satisfies
\begin{equation}
\frac{\dd I}{\dd t}= -\gamma I,\qquad I(0)=N,
\label{eq:recovery}
\end{equation}
with solution $I(t)=N e^{-\gamma t}$.

\subsection*{(b) Mean infectious period and estimate of $\gamma$}
The mean time until recovery is the expected waiting time of an exponential distribution,
\[
\boxed{\;T_\text{inf}=\tfrac1\gamma\;}.\]
For an illness with average infectious period $T_\text{inf}=\SI{5}{day}$ we obtain $\gamma\approx0.20\;\text{day}^{-1}$.

% -------------------------------------------------------------
\section{Problem~3: Infection and Recovery (The SIR Core)}
We now combine infection (Problem~1) with recovery (Problem~2).

\subsection*{(a) Model equations}
\begin{align}
\frac{\dd S}{\dd t} &= -\frac{\beta}{N}\,S I,\\[4pt]
\frac{\dd I}{\dd t} &= \frac{\beta}{N}\, S I-\gamma I.
\label{eq:SIR}
\end{align}
Although the removed class $R$ is not explicit here, one recovers the full SIR system via $R=N-S-I$.

\subsection*{(b) Non--dimensionalisation}
Introduce dimensionless time $\tau=\gamma t$ and population fractions
\[
s(\tau)=\frac{S(t)}{N},\qquad i(\tau)=\frac{I(t)}{N},\qquad s+i=1.
\]
Writing $\tfrac{\dd}{\dd \tau}=\tfrac1\gamma\tfrac{\dd}{\dd t}$, system~\eqref{eq:SIR} becomes
\begin{align}
\frac{\dd s}{\dd \tau} &= -\RR\, s i,\\[4pt]
\frac{\dd i}{\dd \tau} &= \RR\, s i - i,
\label{eq:nondim}
\end{align}
where the (now population--independent) basic reproduction number is
\[
\boxed{\;\RR = \frac{\beta}{\gamma}\;}.\]

\subsection*{(c) Biological interpretation of $\RR$}
$\RR$ equals the expected number of secondary infections generated by a single typical infectious individual introduced into a wholly susceptible population.  An epidemic outbreak requires $\RR>1$.


\end{document}